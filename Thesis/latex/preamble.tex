% --- comeca preambulo ---
\documentclass[
	% -- op\c{c}\~{o}es da classe memoir --
	12pt,				% tamanho da fonte
	oneside,			% para impress\~{a}o em recto e verso. Oposto a oneside
	a4paper,			% tamanho do papel.
	% -- op\c{c}\~{o}es da classe abntex2 --
	%chapter=TITLE,		% t\'{\i}tulos de cap\'{\i}tulos convertidos em letras mai\'{u}sculas
	%section=TITLE,		% t\'{\i}tulos de se\c{c}\~{o}es convertidos em letras mai\'{u}sculas
	%subsection=TITLE,	% t\'{\i}tulos de subse\c{c}\~{o}es convertidos em letras mai\'{u}sculas
	%subsubsection=TITLE,% t\'{\i}tulos de subsubse\c{c}\~{o}es convertidos em letras mai\'{u}sculas
	% -- op\c{c}\~{o}es do pacote babel --
	english,			% idioma adicional para hifeniza\c{c}\~{a}o
	french,				% idioma adicional para hifeniza\c{c}\~{a}o
	spanish,			% idioma adicional para hifeniza\c{c}\~{a}o
	brazil				% o \'{u}ltimo idioma \'{e} o principal do documento
	]{abntex2}

% ---
% Pacotes b\'{a}sicos
% ---
\usepackage{lmodern}			% Usa a fonte Latin Modern			
\usepackage[T1]{fontenc}		% Selecao de codigos de fonte.
\usepackage[utf8]{inputenc}		% Codificacao do documento (convers\~{a}o autom\'{a}tica dos acentos)
\usepackage{indentfirst}		% Indenta o primeiro par\'{a}grafo de cada se\c{c}\~{a}o.
\usepackage{color}				% Controle das cores
\usepackage[pdftex]{graphicx}	% Inclus\~{a}o de gr\'{a}ficos
\usepackage{microtype} 			% para melhorias de justifica\c{c}\~{a}o
%\usepackage{cmap}				% Mapear caracteres especiais no PDF
\usepackage{lastpage}			% Usado pela Ficha catalogr\'{a}fica
\usepackage{epstopdf}           % Pacote que converte as figuras em eps para pdf
% ---
		
% ---
% Pacotes adicionais, usados apenas no \^{a}mbito do Modelo Can\^{o}nico do abnteX2
% ---
\usepackage{lipsum}				% para gera\c{c}\~{a}o de dummy text
\usepackage{nomencl}
\usepackage{amsmath}
\usepackage{bbm}
\usepackage{multirow}
\usepackage{rotating}
\usepackage{pdfpages}
\usepackage[font=footnotesize]{subfig}
\usepackage{booktabs}
\usepackage{pdflscape}
\usepackage{chngcntr}

\let\printglossary\relax
\let\theglossary\relax
\let\endtheglossary\relax

\usepackage[nonumberlist,acronym,nomain]{glossaries} % nonnumberlist nao mostra as paginas nas quais os acronimos aparecem no texto
\newglossary[tlg]{simbolos}{tld}{tdn}{Lista de símbolos}
% Generate the glossary
\makeglossaries

\counterwithin{figure}{chapter}
\counterwithin{table}{chapter}

% ---


% ---
% Pacotes de cita\c{c}\~{o}es
% ---
\usepackage[brazilian,hyperpageref]{backref}	 % Paginas com as cita\c{c}\~{o}es na bibl
\usepackage[alf]{abntex2cite}	% Cita\c{c}\~{o}es padr\~{a}o ABNT

% --- Pacote de customiza\c{c}\~{a}o - Unicamp ---
\usepackage{unicamp}

% ---
% CONFIGURA\c{C}\~{O}ES DE PACOTES
% ---

% ---
% Configura\c{c}\~{o}es do pacote backref
% Usado sem a op\c{c}\~{a}o hyperpageref de backref
\renewcommand{\backrefpagesname}{Citado na(s) p\'{a}gina(s):~}
% Texto padr\~{a}o antes do n\'{u}mero das p\'{a}ginas
\renewcommand{\backref}{}
% Define os textos da cita\c{c}\~{a}o
\renewcommand*{\backrefalt}[4]{
	\ifcase #1 %
		Nenhuma cita\c{c}\~{a}o no texto.%
	\or
		Citado na p\'{a}gina #2.%
	\else
		Citado #1 vezes nas p\'{a}ginas #2.%
	\fi}%
% ---

%\graphicspath{{./eps/}}
%\DeclareGraphicsExtensions{.eps}


% ---
% Configura\c{c}\~{o}es de apar\^{e}ncia do PDF final

% alterando o aspecto da cor azul
\definecolor{blue}{RGB}{41,5,195}

% informa\c{c}\~{o}es do PDF
\makeatletter
\hypersetup{
     	%pagebackref=true,
		pdftitle={\@title},
		pdfauthor={\@author},
    	pdfsubject={\imprimirpreambulo},
	    pdfcreator={LaTeX with abnTeX2},
		pdfkeywords={abnt}{latex}{abntex}{abntex2}{trabalho acad\^{e}mico},
		colorlinks=true,       		% false: boxed links; true: colored links
    	linkcolor=blue,          	% color of internal links
    	citecolor=blue,        		% color of links to bibliography
    	filecolor=magenta,      		% color of file links
		urlcolor=blue,
		bookmarksdepth=4
}
\makeatother
% ---

% ---
% Posiciona figuras e tabelas no topo da p\'{a}gina quando adicionadas sozinhas
% em um p\'{a}gina em branco. Ver https://github.com/abntex/abntex2/issues/170
\makeatletter
\setlength{\@fptop}{5pt} % Set distance from top of page to first float
\makeatother
% ---

% ---
% Possibilita cria\c{c}\~{a}o de Quadros e Lista de quadros.
% Ver https://github.com/abntex/abntex2/issues/176
%
\newcommand{\quadroname}{Quadro}
\newcommand{\listofquadrosname}{Lista de quadros}

\newfloat[chapter]{quadro}{loq}{\quadroname}
\newlistof{listofquadros}{loq}{\listofquadrosname}
\newlistentry{quadro}{loq}{0}

% configura\c{c}\~{o}es para atender \`{a}s regras da ABNT
\setfloatadjustment{quadro}{\centering}
\counterwithout{quadro}{chapter}
\renewcommand{\cftquadroname}{\quadroname\space}
\renewcommand*{\cftquadroaftersnum}{\hfill--\hfill}

\setfloatlocations{quadro}{hbtp} % Ver https://github.com/abntex/abntex2/issues/176
% ---

% ---
% Espa\c{c}amentos entre linhas e par\'{a}grafos
% ---

% O tamanho do par\'{a}grafo \'{e} dado por:
\setlength{\parindent}{2cm}

% Controle do espa\c{c}amento entre um par\'{a}grafo e outro:
\setlength{\parskip}{0.2cm}  % tente tamb\'{e}m \onelineskip





% ---
% Informa\c{c}\~{o}es de dados para CAPA e FOLHA DE ROSTO
% ---
\titulo{Dissertation/Thesis Title}
\autor{Author Name}
\local{Campinas}
\data{Year}
\orientador{Prof. Dr. Supervisor}
\coorientador[Coorientador]{Prof. Dr. Co-supervisor}
\instituicao{%
  UNIVERSIDADE ESTADUAL DE CAMPINAS
    \par
    Faculdade de Engenharia Elétrica e de Computação
    }

% MSc
%\tipotrabalho{Dissertation (Masters)}
%\preambulo{Dissertation presented to the School of Electrical and Computer Engineering of the University of Campinas in partial fulfillment of the requirements for the degree of Master in Electrical Engineering, in the area of Computer Engineering.}

% PhD
\tipotrabalho{Thesis (Doctorate)}
\preambulo{Thesis presented to the School of Electrical and Computer Engineering of the University of Campinas in partial fulfillment of the requirements for the degree of Doctor in Electrical Engineering, in the area of Computer Engineering.}
% ---

% --- finaliza preambulo ---
