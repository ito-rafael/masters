% Created 2022-10-06 Thu 17:42
% Intended LaTeX compiler: pdflatex
\documentclass[11pt]{article}
\usepackage[utf8]{inputenc}
\usepackage[T1]{fontenc}
\usepackage{graphicx}
\usepackage{longtable}
\usepackage{wrapfig}
\usepackage{rotating}
\usepackage[normalem]{ulem}
\usepackage{amsmath}
\usepackage{amssymb}
\usepackage{capt-of}
\usepackage{hyperref}
\author{John Doe}
\date{\today}
\title{}
\hypersetup{
 pdfauthor={John Doe},
 pdftitle={},
 pdfkeywords={},
 pdfsubject={},
 pdfcreator={Emacs 28.2 (Org mode 9.6)}, 
 pdflang={English}}
\begin{document}

\tableofcontents

\imprimircapa


\imprimirfolhaderosto*


\begin{fichacatalografica}
    \vspace*{\fill}
    \begin{center}
        \textsc{Inclua aqui o pdf com a ficha catalogr\'{a}fica fornecida pela BAE.}
    \end{center}
    \vspace*{\fill}
    %\includepdf{fig_ficha_catalografica.pdf}
\end{fichacatalografica}


\begin{folhadeaprovacao}
  \begin{center}
    COMISS\~{A}O JULGADORA - TESE DE DOUTORADO
    %\textsc{Inclua aqui a folha de assinaturas.}
\end{center}
\noindent
\begin{minipage}{\textwidth}\SingleSpacing
Candidato(a): Nome do Autor      RA: XXXXXX
Data de defesa: XX de MES de 202X
Título da Tese: "XXXXXXXXXXXXXXXXXXXXXXXXXXXXXXX"
\vspace{2cm}

Profa. Dra. Xxxxxxxxxx (Presidente)

Profa. Dra. xxxxxxx

Profa. Dra. xxxxxxx

Profa. Dra. xxxxxxxxx

Profa. Dra xxxxxxxxxxxx

\vspace{2cm}

A Ata de Defesa, com as respectivas assinaturas dos membros da Comiss\~{a}o Julgadora, encontra-se no SIGA (Sistema de Fluxo de Disserta\c{c}\~{a}o/Tese) e na Secretaria de P\'{o}s-Gradua\c{c}\~{a}o da Faculdade de Engenharia El\'{e}trica e de Computa\c{c}\~{a}o.
\end{minipage}

\end{folhadeaprovacao}


\begin{dedicatoria}
   \vspace*{\fill}
   \centering
   \noindent
   \textit{ Dedico esta tese \`{a} todo mundo.} \vspace*{\fill}
\end{dedicatoria}


\begin{agradecimentos}
    Escreva seus agradecimentos.
    Obs.: Ser\'{a} obrigat\'{o}rio caso o autor tenha recebido aux\'{\i}lio financeiro, parcial ou integral, de ag\^{e}ncia (s) de fomento. Neste caso, ele dever\'{a} atender \`{a} legisla\c{c}\~{a}o vigente espec\'{\i}fica de cada uma das Ag\^{e}ncias quanto \`{a} necessidade de se fazer refer\^{e}ncia ao apoio recebido e ao n\'{u}mero de processo. (OF PRPG 002/2019 – Orienta\c{c}\~{a}o sobre disserta\c{c}\~{o}es e teses).

    -	Em caso de Bolsa CAPES, usar a seguinte express\~{a}o, como indicado na portaria 206 da CAPES:

    O presente trabalho foi realizado com apoio da Coordena\c{c}\~{a}o de Aperfei\c{c}oamento de Pessoal de N\'{\i}vel Superior - Brasil (CAPES) - C\'{o}digo de Financiamento 001.

    -	Em caso de bolsa CNPQ, usar a seguinte express\~{a}o de agradecimento:

    O presente trabalho foi realizado com apoio do CNPq, Conselho Nacional de Desenvolvimento Cient\'{\i}fico  e Tecnol\'{o}gico – Brasil.

    -	Em caso de bolsa FAPESP, fazer agradecimento contendo nome FAPESP, o n\'{u}mero do processo FAPESP a que se refere este Termo de Outorga, no modelo:

    processo nº aaaa/nnnnn-d, Funda\c{c}\~{a}o de Amparo \`{a} Pesquisa do Estado de S\~{a}o Paulo (FAPESP).

    Os artigos escritos em idioma estrangeiro dever\~{a}o indicar o apoio da FAPESP em ingl\^{e}s, conforme o seguinte modelo: grant \# aaaa/nnnnn-d, S\~{a}o Paulo Research Foundation (FAPESP).
\end{agradecimentos}


\begin{epigrafe}
    \vspace*{\fill}
    \begin{flushright}
        \textit{``Escreva aqui a sua ep\'{\i}grafe (Opcional)''\\
        (Cita\c{c}\~{a}o)}
    \end{flushright}
\end{epigrafe}


\% resumo em português
\setlength{\absparsep}{18pt} \% ajusta o espa\c{c}amento dos par$\backslash$'\{a\}grafos do resumo
\begin{resumo}
 Insira seu resumo. (Obrigat\'{o}rio, em portugu\^{e}s m\'{a}ximo de 500 palavras)

	\lipsum[1]

    \vspace{\onelineskip}

    \noindent\textbf{Palavras-chaves}: palavra-chave 1; palavra-chave 2; palavra-chave 3.
\end{resumo}


\% resumo em inglês
\begin{resumo}[Abstract]
 \begin{otherlanguage*}{english}
   Same content of "Resumo".

   \vspace{\onelineskip}

   \noindent
   \textbf{Keywords}: latex. abntex. text editoration.
 \end{otherlanguage*}
\end{resumo}


\% inserir lista de ilustrações
\pdfbookmark[0]{\listfigurename}{lof}
\listoffigures*
\cleardoublepage


\% inserir lista de quadros
\%\pdfbookmark[0]{\listofquadrosname}{loq}
\%\listofquadros*
\%\cleardoublepage


\% inserir lista de tabelas
\pdfbookmark[0]{\listtablename}{lot}
\listoftables*
\cleardoublepage


\% inserir lista de abreviaturas e siglas

\% --- inserir lista de Acronimos e Abrevia\c{c}$\backslash$~\{o\}es ---
\newacronym{NIST}{NIST}{National Institute of Standards and Technology}
\newacronym{TWC}{TWC}{Transformada Wavelet Contínua}
\newacronym{TWD}{TWD}{Transformada Wavelet Discreta}
\printglossary[type=\acronymtype,title=\{List of Abbreviations\}]
\cleardoublepage

\% --- inserir lista de Acronimos e Abrevia\c{c}$\backslash$~\{o\}es ---
\%\newglossaryentry{NN}
{%
	name=$N$x$N$,
	type=simbolos,
	description={Dimensão de uma imagem na forma altura x largura, composta por $N$ x $N$ pixels}
}

\newglossaryentry{NFNF}
{%
	name=$N_{f}$x$N_{f}$,
	type=simbolos,
	description={Dimensão de um filtro de imagem na forma altura x largura (\textit{N\textsubscript{f}} x \textit{N\textsubscript{f}})}
}
\end{document}
