\selectlanguage{english}
\imprimircapa
\imprimirfolhaderosto*

\begin{fichacatalografica}
    \vspace*{\fill}
    \begin{center}
        \textsc{Inclua aqui o pdf com a ficha catalográfica fornecida pela BAE.}
    \end{center}
    \vspace*{\fill}
    %\includepdf{fig_ficha_catalografica.pdf}
\end{fichacatalografica}

\begin{folhadeaprovacao}
  \begin{center}
    COMISSÃO JULGADORA - TESE DE DOUTORADO
    %\textsc{Inclua aqui a folha de assinaturas.}
\end{center}
\noindent
\begin{minipage}{\textwidth}\SingleSpacing
Candidato(a): Nome do Autor      RA: XXXXXX
Data de defesa: XX de MES de 202X
Título da Tese: "XXXXXXXXXXXXXXXXXXXXXXXXXXXXXXX"
\vspace{2cm}

Profa. Dra. Xxxxxxxxxx (Presidente)

Profa. Dra. xxxxxxx

Profa. Dra. xxxxxxx

Profa. Dra. xxxxxxxxx

Profa. Dra xxxxxxxxxxxx

\vspace{2cm}

A Ata de Defesa, com as respectivas assinaturas dos membros da Comissão Julgadora, encontra-se no SIGA (Sistema de Fluxo de Dissertação/Tese) e na Secretaria de Pós-Graduação da Faculdade de Engenharia Elétrica e de Computação.
\end{minipage}

\end{folhadeaprovacao}

\begin{dedicatoria}
   \vspace*{\fill}
   \centering
   \noindent
   \textit{ Dedico esta tese à todo mundo.} \vspace*{\fill}
\end{dedicatoria}

\begin{agradecimentos}
    Escreva seus agradecimentos.
    Obs.: Será obrigatório caso o autor tenha recebido auxílio financeiro, parcial ou integral, de agência (s) de fomento. Neste caso, ele deverá atender à legislação vigente específica de cada uma das Agências quanto à necessidade de se fazer referência ao apoio recebido e ao número de processo. (OF PRPG 002/2019 – Orientação sobre dissertações e teses).

    -	Em caso de Bolsa CAPES, usar a seguinte expressão, como indicado na portaria 206 da CAPES:

    O presente trabalho foi realizado com apoio da Coordenação de Aperfeiçoamento de Pessoal de Nível Superior - Brasil (CAPES) - Código de Financiamento 001.

    -	Em caso de bolsa CNPQ, usar a seguinte expressão de agradecimento:

    O presente trabalho foi realizado com apoio do CNPq, Conselho Nacional de Desenvolvimento Científico  e Tecnológico – Brasil.

    -	Em caso de bolsa FAPESP, fazer agradecimento contendo nome FAPESP, o número do processo FAPESP a que se refere este Termo de Outorga, no modelo:

    processo nº aaaa/nnnnn-d, Fundação de Amparo à Pesquisa do Estado de São Paulo (FAPESP).

    Os artigos escritos em idioma estrangeiro deverão indicar o apoio da FAPESP em inglês, conforme o seguinte modelo: grant \# aaaa/nnnnn-d, São Paulo Research Foundation (FAPESP).
\end{agradecimentos}

\begin{epigrafe}
    \vspace*{\fill}
    \begin{flushright}
        \textit{``Escreva aqui a sua epígrafe (Opcional)''\\
        (Citação)}
    \end{flushright}
\end{epigrafe}

\begin{resumo}[Abstract]
\begin{otherlanguage*}{english}

    Same content of "Resumo".

    \lipsum[1]

    \vspace{\onelineskip}

    \noindent
    \textbf{Keywords}:
        latex.
        abntex.
        text editoration.
\end{otherlanguage*}
\end{resumo}

\setlength{\absparsep}{18pt}
\begin{resumo}[Resumo]

    Insira seu resumo. (Obrigatório, em português máximo de 500 palavras)

    \lipsum[1]

    \vspace{\onelineskip}

    \noindent\textbf{Palavras-chaves}:
        palavra-chave 1;
        palavra-chave 2;
        palavra-chave 3.
\end{resumo}

\pdfbookmark[0]{\listfigurename}{lof}
\listoffigures*
\cleardoublepage

%\pdfbookmark[0]{\listofquadrosname}{loq}
%\listofquadros*
%\cleardoublepage

\pdfbookmark[0]{\listtablename}{lot}
\listoftables*
\cleardoublepage

\newacronym{NIST}{NIST}{National Institute of Standards and Technology}
\newacronym{TWC}{TWC}{Transformada Wavelet Contínua}
\newacronym{TWD}{TWD}{Transformada Wavelet Discreta}
\printglossary[type=\acronymtype,title={List of Abbreviations}]
\cleardoublepage

%\newglossaryentry{NN}
{%
	name=$N$x$N$,
	type=simbolos,
	description={Dimensão de uma imagem na forma altura x largura, composta por $N$ x $N$ pixels}
}

\newglossaryentry{NFNF}
{%
	name=$N_{f}$x$N_{f}$,
	type=simbolos,
	description={Dimensão de um filtro de imagem na forma altura x largura (\textit{N\textsubscript{f}} x \textit{N\textsubscript{f}})}
}

%\newglossaryentry{NN}{
%    name=$N$x$N$,
%    type=simbolos,
%    description={Dimensão de uma imagem na forma altura x largura, composta por $N$ x $N$ pixels}
%}
%
%\newglossaryentry{NFNF}{
%    name=$N_{f}$x$N_{f}$,
%    type=simbolos,
%    description={Dimensão de um filtro de imagem na forma altura x largura (\textit{N\textsubscript{f}} x \textit{N\textsubscript{f}})}
%}

\printglossary[type=simbolos,title={List of Symbols}]
\cleardoublepage

\pdfbookmark[0]{\contentsname}{toc}
\tableofcontents*
\cleardoublepage

\textual
